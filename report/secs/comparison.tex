\section{Comparison between Part I and Part II}

In this section I will point up the gap between the \cplex{} solver and the
genetic algorithm I implemented.

The scenarios I used to make this comparison are the same ones against which I
benchmarked Part I and Part II programs.

\subsection{Gap with the optimal solution}

The first comparison I will show is the one in which fixed instances were used.
Obviously, since the genetic algorithm always managed to find the optimal
solution for the \textit{345} instance, we have no loss of precision in that
situation.

\paragraph{Note} The values in the table are the mean percentage of error from
the optimal solution.

\begin{table}[H]
  \centering
  \begin{tabular}{|l|r|r|r|r|r|}
    \hline
    \textbf{Instance} & \textbf{Avg} & \textbf{BE} &
    \textbf{Mut} & \textbf{NH} & \textbf{Time} \\
    \hline
    \hline
    \textit{345} & $0\%$ & $0\%$ & $0\%$ & $0\%$ & $0\%$ \\
    \hline
    \textit{tsp12} & $38.75\%$ & $36.92\%$ &
            $39.14\%$ & $43.50\%$ & $8.69\%$ \\
    \hline
    \textit{tsp60} & $324.74\%$ & $301.63\%$ &
            $319.31\%$ & $310.18\%$ & $228.58\%$ \\
    \hline
    \textit{gerber} & $42.73\%$ & $48.43\%$ &
            $56.79\%$ & $47.07\%$ & $10.90\%$ \\
    \hline
  \end{tabular}
  \caption{Precision loss for fixed instances with different calibrations}
  \label{tab:precision-fixed}
\end{table}

After showing the results for the fixed instances, I will provide the results
for the randomly generated instances by grouping them by number of holes and
size.

\begin{table}[H]
  \centering
  \begin{tabular}{|l|r|r|r|r|r|}
    \hline
    \textbf{Instance} & \textbf{Avg} & \textbf{BE} &
    \textbf{Mut} & \textbf{NH} & \textbf{Time} \\
    \hline
    \hline
    \textit{ru} -- $10\ 10\ 2$ &
    $28.86\%$ & $35.26\%$ & $47.87\%$ & $45.67\%$ & $12.76\%$ \\
    \hline
    \textit{rg} -- $10\ 10\ 2$ &
    $13.14\%$ & $15.66\%$ & $14.45\%$ & $10.78\%$ & $5.11\%$ \\
    \hline
    \textit{rg\_Manh} -- $10\ 10\ 2$ &
    $29.18\%$ & $27.57\%$ & $26.36\%$ & $28.61\%$ & $4.68\%$ \\
    \hline
    \textit{ru} -- $10\ 10\ 4$ &
    $31.75\%$ & $25.60\%$ & $38.22\%$ & $30.49\%$ & $38.23\%$ \\
    \hline
    \textit{rg} -- $10\ 10\ 4$ &
    $17.14\%$ & $18.80\%$ & $21.43\%$ & $15.60\%$ & $9.42\%$ \\
    \hline
    \textit{rg\_Manh} -- $10\ 10\ 4$ &
    $35.57\%$ & $38.11\%$ & $35.13\%$ & $28.67\%$ & $28.52\%$ \\
    \hline
  \end{tabular}
  \caption{Precision loss for instances with 10 holes and a 10x size factor}
  \label{tab:precision-10-10}
\end{table}

\begin{table}[H]
  \centering
  \begin{tabular}{|l|r|r|r|r|r|}
    \hline
    \textbf{Instance} & \textbf{Avg} & \textbf{BE} &
    \textbf{Mut} & \textbf{NH} & \textbf{Time} \\
    \hline
    \hline
    \textit{ru} -- $10\ 100\ 2$ &
    $52.98\%$ & $53.37\%$ & $43.04\%$ & $43.40\%$ & $18.82\%$ \\
    \hline
    \textit{rg} -- $10\ 100\ 2$ &
    $30.35\%$ & $26.41\%$ & $42.58\%$ & $33.94\%$ & $13.37\%$ \\
    \hline
    \textit{rg\_Manh} -- $10\ 100\ 2$ &
    $40.72\%$ & $34.01\%$ & $38.22\%$ & $35.04\%$ & $21.42\%$ \\
    \hline
    \textit{ru} -- $10\ 100\ 4$ &
    $33.13\%$ & $32.23\%$ & $33.57\%$ & $17.34\%$ & $13.45\%$ \\
    \hline
    \textit{rg} -- $10\ 100\ 4$ &
    $24.38\%$ & $21.32\%$ & $23.03\%$ & $21.16\%$ & $4.82\%$ \\
    \hline
    \textit{rg\_Manh} -- $10\ 100\ 4$ &
    $22.52\%$ & $23.85\%$ & $27.17\%$ & $22.98\%$ & $5.05\%$ \\
    \hline
  \end{tabular}
  \caption{Precision loss for instances with 10 holes and a 100x size factor}
  \label{tab:precision-10-100}
\end{table}

\begin{table}[H]
  \centering
  \begin{tabular}{|l|r|r|r|r|r|}
    \hline
    \textbf{Instance} & \textbf{Avg} & \textbf{BE} &
    \textbf{Mut} & \textbf{NH} & \textbf{Time} \\
    \hline
    \hline
    \textit{ru} -- $50\ 10\ 2$ &
    $880.64\%$ & $855.08\%$ & $879.58\%$ & $887.82\%$ & $539.65\%$ \\
    \hline
    \textit{rg} -- $50\ 10\ 2$ &
    $143.71\%$ & $147.63\%$ & $137.85\%$ & $143.58\%$ & $111.20\%$ \\
    \hline
    \textit{rg\_Manh} -- $50\ 10\ 2$ &
    $151.96\%$ & $145.27\%$ & $148.62\%$ & $151.06\%$ & $119.41\%$ \\
    \hline
    \textit{ru} -- $50\ 10\ 4$ &
    $835.40\%$ & $844.56\%$ & $888.89\%$ & $864.58\%$ & $549.03\%$ \\
    \hline
    \textit{rg} -- $50\ 10\ 4$ &
    $163.71\%$ & $157.69\%$ & $152.06\%$ & $144.95\%$ & $111.92\%$ \\
    \hline
    \textit{rg\_Manh} -- $50\ 10\ 4$ &
    $139.22\%$ & $138.87\%$ & $153.23\%$ & $154.78\%$ & $118.00\%$ \\
    \hline
  \end{tabular}
  \caption{Precision loss for instances with 50 holes and a 10x size factor}
  \label{tab:precision-50-10}
\end{table}

\begin{table}[H]
  \centering
  \begin{tabular}{|l|r|r|r|r|r|}
    \hline
    \textbf{Instance} & \textbf{Avg} & \textbf{BE} &
    \textbf{Mut} & \textbf{NH} & \textbf{Time} \\
    \hline
    \hline
    \textit{ru} -- $50\ 100\ 2$ &
    $680.37\%$ & $674.16\%$ & $685.20\%$ & $701.89\%$ & $458.52\%$ \\
    \hline
    \textit{rg} -- $50\ 100\ 2$ &
    $157.00\%$ & $160.94\%$ & $162.68\%$ & $160.37\%$ & $134.29\%$ \\
    \hline
    \textit{rg\_Manh} -- $50\ 100\ 2$ &
    $179.84\%$ & $166.84\%$ & $176.30\%$ & $174.69\%$ & $140.24\%$ \\
    \hline
    \textit{ru} -- $50\ 100\ 4$ &
    $671.49\%$ & $667.12\%$ & $696.45\%$ & $723.28\%$ & $448.72\%$ \\
    \hline
    \textit{rg} -- $50\ 100\ 4$ &
    $165.13\%$ & $157.54\%$ & $177.00\%$ & $150.95\%$ & $122.31\%$ \\
    \hline
    \textit{rg\_Manh} -- $50\ 100\ 4$ &
    $158.09\%$ & $153.80\%$ & $165.80\%$ & $145.64\%$ & $118.32\%$ \\
    \hline
  \end{tabular}
  \caption{Precision loss for instances with 50 holes and a 100x size factor}
  \label{tab:precision-50-100}
\end{table}

We can obtain some insights from this data:

\begin{itemize}
\item Here we have a further proof of what I stated in the previous section:
  the genetic algorithm is not able to cope with big instances and short
  amounts of time. In fact, when dealing with 50 or 60 holes, the algorithm
  was not able to produce (on average) a solution which has an error lower
  than $100\%$;

\item After further tests\footnote{not reported in this document because they
  were done during the phase of parameter tuning} I realized that even pushing
  the timeout to three or four minutes, the precision we gain is not valuable
  as it is switching from say, 5 seconds to a minute.

\item At the same time, we can notice that the performance of the genetic
  algorithm on small instances\footnote{instances with $\leq\ 15$ holes} are
  not bad, since they on average get wrong under the $50\%$ w.r.t. optimality
  almost on all cases.

\item Another observation we can make is that the genetic algorithm probably
  works better with more realistic instances: in fact, using the Random Uniform
  generation we definitely have the worst scores, even when comparing them with
  the 60-holes instance.
\end{itemize}
