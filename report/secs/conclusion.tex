\section{Conclusions}\label{sec:conclusions}

The tests revealed that some instances work way more worse than others: in
particular, Random Uniform generated instances (i.e., the ones generated with
true randomness) present bad features:

\begin{itemize}
  \item they do not provide realistic data; and
  \item the genetic algorithm performs particularly poorly when run against one
    RU instance.
\end{itemize}

Aside from that, both the exact and the heuristic solver ought to be run against
a Gerber instance, if there is one available for the specific problem.

A minor feature that was not developed was integrating the Manhattan distance in
instances other than the Random Grid Uniform. In fact, I did not think that it
would have yielded so much added value since you can easily add it to any of the
generators (even the Gerber one) if you ever had the need. \\

Regarding the performance comparison between the two methods, the evaluation of
the trade-off between the efficiency of a genetic algorithm and an exact method
depends on up to loss of precision we are willing to accept an approximated
solution.

In fact, PCBs are much more like Random Grid instances and do not have a large
number of holes: in fact, if electronic boards producers want to have a huge
number of components (and therefore of holes/pins) on their cards, they probably
would use SSIs, MSIs, LSIs or VLSIs\footnote{the number of components in this
case range from about some tens to billions of elements}.

Therefore, the genetic algorithm does not perform that bad when considering a
realistic use case in which it has to design the route for a driller that has
the task to make the holes in the PCB. \\

Also, the genetic algorithm is free of charge, whereas the \cplex{} solver would
cost about \$$10000$ for a single machine, so someone who wants to compute the
optimal path for the driller may use the heuristic method having a decent
trade-off also considering the pricing aspect.

A probable threshold point would be if this person has to produce a number of
boards such that the profits would outnumber \$$10000$ if he was able to
improve by a factor of 1.5x or 2x the speed of the production.
