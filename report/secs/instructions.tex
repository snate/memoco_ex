\section{Instructions for use}\label{sec:instructions}

If you want to use this program, you can refer to the \texttt{README.md} present
in the root folder and the other ones in the \texttt{ex1\_drilling} and
\texttt{ex2\_heur} folders.

I'll anyway put some basic instructions also in this document to provide a
minimal reference for the use of the whole program.

\subsection{Generating the gerber instances}

In order to generate instances starting from Gerber files, open the file
\texttt{ex1\_drilling/gen/gerber.html} in a browser.

This webpage works both as a parser and a viewer for \texttt{.gbr} files; more
detailed instructions are present in the \texttt{README.md} in the
\texttt{ex1\_drilling} folder.

\subsection{Generating the random instances}

In order to generate some random instances, you have to move to the
\texttt{ex1\_drilling} folder. Although the generators' source code is placed
in the \texttt{gen} subfolder, I prepared some simple bash scripts for an easier
usage.

You can generate an instance by simply entering the following command in a bash
shell:

\begin{verbatim}
bash scripts/generation/generateRandomData_<instance>.sh <no_of_holes> <board_size_factor> <divisions>
\end{verbatim}

You can choose an instance among the one listed in the \texttt{README.md} file.
Also, the README explains what are the three parameters present in the above
command.

\subsection{Running the exact method}

If you want to run the program that uses the \cplex{} solver, move to the
\texttt{ex1\_drilling} folder. From here you can run some scripts to run the
program by typing:

\begin{verbatim}
bash scripts/run/run_<instance>.sh
\end{verbatim}

Where $<$\textit{instance}$>$ is the type of file you are willing to run the
solver against.

\subsection{Running the genetic algorithm}

If you want to run the program that uses the genetic algorithm, move to the
\texttt{ex2\_heur} folder. From here you can run some scripts to run the
program by typing:

\begin{verbatim}
bash scripts/run/run_<instance>.sh <configuration>
\end{verbatim}

Where $<$\textit{instance}$>$ is the type of file you are willing to run the
algorithm against and $<$\textit{configuration}$>$ is the file containing the
calibration for the genetic algorithm parameters.

You are free to write your own configuration (tuning) file and run the genetic
algorithm using it for its settings.

More details are present in the \texttt{README.md} file in the
\texttt{ex2\_heur} folder.
